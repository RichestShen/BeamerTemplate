\documentclass[aspectratio=169]{beamer}  % 长宽比

%% 导入宏包
\usepackage[UTF8, noindent]{ctexcap}  % 中文支持
\usepackage{amsfonts, amsmath}  % 数学支持
\usepackage{verbatim}  % 用于原样显示,常用于插入行内代码
\usepackage{listings}  % 插入代码并显支持代码高亮
\usepackage{graphicx, wrapfig}  % 图文混排包
\usepackage{algorithm, algorithmic}  % 伪代码

%% 样式主题
\usetheme{Madrid}
\usecolortheme{default}
\bibliographystyle{IEEEtrans}  % 参考文献样式

%% 封面页信息
\title{Beamer 幻灯片制作极简入门}
% \subtitle{}
\author[Q.Tom, x.Jerry] % 方括号内填写姓名简称, 可省略
{Quattro Tom\inst{1} \and xDrive Jerry\inst{2}}
\institute[TJU] % 方括号内填写机构简称, 可省略
{
  \inst{1}
  College of Electronic and Information Engineering\\
  Tongji University
  \and
  \inst{2}
  School of Automotive Studies\\
  Tongji University
}
\date{28 Aug, 2024}
\logo{\includegraphics[scale=0.15]{./src/logo}}
\definecolor{uoftblue}{RGB}{6,41,88}
\setbeamercolor{titlelike}{bg=uoftblue}
\setbeamerfont{title}{series=\bfseries}

%% 正文
\begin{document}
\frame{\titlepage}

\begin{frame}
\frametitle{目录}
    \tableofcontents
\end{frame}

\section{组织 Beamer 内容}
\subsection{导言区设置}
\begin{frame}{导言区}
    如图所示, 设置导言区常用配置
    \begin{figure}[htbp]
        \centering
        \begin{minipage}[t]{0.48\textwidth}
            \centering
            \includegraphics[scale=0.4]{./src/frame1.png}
            \caption{常用宏包}
        \end{minipage}
        \begin{minipage}[t]{0.48\textwidth}
            \centering
            \includegraphics[scale=0.4]{./src/frame2.png}
            \caption{封面页信息}
        \end{minipage}
    \end{figure}
\end{frame}

\subsection{基本单元--frame}
\begin{frame}{帧(frame)}
    \begin{itemize}
        \item Beamer 中,用 frame 环境得到帧, 默认整体垂直居中显示.
        \item 在 frame 后面, 用 \{\} 添加帧的标题, 可选.
        \item 可以使用 \LaTeX{} 定理类环境.
        \item 默认帧尺寸为 4:3, 若要更改, 在 documentclass 后加入 [aspectratio=169], 更改为 16:9.
    \end{itemize}
\end{frame}

\subsection{目录}
\begin{frame}{分节 \& 目录}
    \begin{itemize}
        \item 使用 part, section, subsection 分节, 不建议使用更深的目录层次.
        \item 使用 tableofcontents 命令生成目录, 注意该命令必须在单独的帧内使用.
        \item 使用 partpage 生成标题页, 用法类似.
    \end{itemize}
\end{frame}

\section{常用环境}
\subsection{定理类环境}
\begin{frame}{定理 \& 区块}
    \begin{itemize}
        \item Beamer 中预定义了 theorem, corollary, definition, definitions, fact, example, examples 环境.
        \item 可以使用 newtheroem 自定义中文定理环境.
        \item 特别地, ctex 已经将上述环境重定义为中文, 不建议再自行定义.
        \item proof 环境汉化
        \item 区块: block, alertblock, exampleblock
    \end{itemize}
\end{frame}

\begin{frame}{定理环境举例}
	\begin{definition}
		若随机变量 $X$ 服从正态分布, 则其概率密度函数为
		\begin{align}
			f(x)=\frac{1}{\sigma \sqrt{2 \pi}} e^{-\frac{(, t-\mu)^{2}}{2 \sigma^{2}}},
		\end{align}
		记作
		\begin{align}
			X \sim N(\mu, \sigma^2).
		\end{align}
	\end{definition}
	% \renewcommand{\proofname}{证}
	\begin{proof}
		显然成立, QED.
	\end{proof}
\end{frame}

\subsection{图表排版}
\begin{frame}{图表}
	Beamer 中 figure 和 table 不再是浮动体.
	\begin{figure}
		\centering
		\includegraphics[scale=0.05]{./src/demo1.jpg}
		\caption{示例图片}
	\end{figure}
	\begin{table}[!ht]
		\centering
		\begin{tabular}{|l|l|l|}
			\hline
			Mercedes & S 450 & 3.0T \\ \hline
			BMW & M760 Li & 4.4T \\ \hline
			Audi & Rs 6 Avant & 4.0T \\ \hline
		\end{tabular}
		\caption{示例表格}
	\end{table}
\end{frame}

\begin{frame}{多图并排}
    \begin{figure}[htbp]
        \centering
        \begin{minipage}[t]{0.48\textwidth}
            \centering
            \includegraphics[scale=0.08]{./src/demo2.jpg}
            \caption{Mercedes}
        \end{minipage}
        \begin{minipage}[t]{0.48\textwidth}
            \centering
            \includegraphics[scale=0.08]{./src/demo2.jpg}
            \caption{AMG}
        \end{minipage}
    \end{figure}
\end{frame}

\begin{frame}{图文混排}
    \begin{wrapfigure}{l}{0.6\textwidth}
        \centering
        \includegraphics[scale=0.08]{./src/demo3.jpg}
    \end{wrapfigure}
    需添加 wrapfig, graphicx 宏包
    
    `l/r' 代表左侧或右侧

    注意调节左右占比和图片大小, 往往需要反复调节

\end{frame}

\subsection{伪代码}
\begin{frame}{伪代码}
    \begin{algorithm}[H]  % 使用浮动体, 否则引起报错
    \caption{Adam Optimizer} 
    \label{alg1}
    \begin{algorithmic}
        \REQUIRE{step size $\alpha$, decay rates $\beta_1, \beta_2$, stochastic objective function $f(\theta)$, initial paramater $\theta_0$}
        \STATE{$m_0 \rightarrow 0, v_0 \rightarrow 0, t \rightarrow 0$}
        \WHILE{$\theta_t$ not converged}
            \STATE{$t \rightarrow t + 1$}
            \STATE{$g_t \rightarrow \nabla_\theta$}
            \STATE{...}
        \ENDWHILE
        \RETURN{$\theta_t$}
    \end{algorithmic}
    \end{algorithm}
\end{frame}

\section{幻灯片外观}
\subsection{使用主题}
\begin{frame}{使用主题}
	\begin{itemize}
		\item 可以使用 Beamer 预定义的\href{http://latex.artikel-namsu.de/english/themes/uebersicht_beamer.html}{幻灯片主题}, 在导言区用 usetheme 命令载入.
		\item 使用上述命令, 本质上是载入了 .sty 格式文件, 可以自行编写挂载主题文件.
		\item 使用主题后, 仍可以单独配置部分设置.
		\item 注意 ``内容与格式相分离'' 的概念, 在正文中, 尽量只写与文档逻辑结构有关的命令.
	\end{itemize}
\end{frame}
\subsection{动态效果}
\begin{frame}{动态效果}
	\begin{itemize}
		\item 使用 pause 命令, 实现逐步显示.
		\pause
		\item 在 tableofcontents 后加上 pausesections 选项, 使目录每项依次展示.
		\pause
		\item 使用 onslide 指定内容在第几步显示.
	\end{itemize}
\end{frame}

\subsection{bib 引用}
\begin{frame}{插入引用}
    \begin{itemize}
        \item 与 \LaTeX 文档类似, 采用 BibLaTex\cite{ni2019performance} \pause
        \item 在参考文献较多时, 可考虑采用 allowframebreaks 命令允许分页\cite{ni2021multi,zhang2020model} \pause
        \item 效果如末尾页所示
    \end{itemize}
\end{frame}

\begin{frame}{References} 
    \bibliography{ref}
\end{frame}

\end{document}

% \begin{frame}{}
    
% \end{frame}